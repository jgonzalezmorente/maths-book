\documentclass[a4paper,12pt,oneside]{book} % Define el tipo de documento como libro, en papel A4, con tamaño de fuente 12pt y diseño a una sola cara.

% Configuración de codificación y fuentes
\usepackage[utf8]{inputenc}      % Permite usar caracteres UTF-8 en el archivo fuente.
\usepackage[T1]{fontenc}         % Usa codificación de fuente T1 para caracteres extendidos.
\usepackage{lmodern}             % Fuente moderna de LaTeX que mejora la visualización en PDF.
\usepackage[spanish]{babel}      % Configuración para español (división silábica y traducción de elementos como "Capítulo").

% Paquetes matemáticos
\usepackage{amsmath, amsthm, amssymb, amsfonts, calrsfs} % Paquetes para símbolos matemáticos, teoremas, y fuentes adicionales.

% Configuración de márgenes
\usepackage[a4paper, margin=2.5cm]{geometry} % Define los márgenes del documento (2.5 cm en este caso).

% Encabezados y pies de página
\usepackage{fancyhdr}            % Permite personalizar encabezados y pies de página.

% Enlaces y referencias
\usepackage[hidelinks]{hyperref} % Habilita enlaces en el PDF ocultando los colores.

% Configuración de espaciado
\usepackage{setspace}            % Permite ajustar el espaciado entre líneas.
\onehalfspacing                  % Establece espaciado de 1.5 líneas en todo el documento.

% Creación de gráficos vectoriales
%\usepackage{tikz}

% Configuración de encabezados
\setlength{\headheight}{15.5pt}  % Ajusta la altura del encabezado para evitar advertencias.
\pagestyle{fancy}                % Activa el estilo de página personalizado con "fancyhdr".
\fancyhead{}                     % Limpia los encabezados predeterminados.
\fancyhead[C]{\leftmark}         % Muestra el nombre del capítulo centrado en el encabezado.
\fancyfoot{}                     % Limpia los pies de página predeterminados.
\fancyfoot[C]{\thepage}          % Muestra el número de página centrado en el pie de página.

% Definición de entornos para teoremas y otros conceptos matemáticos
\newtheorem{theorem}{Teorema}[chapter]       % Define el entorno "Teorema", numerado por capítulo.
\newtheorem{prop}{Proposición}[chapter]      % Define "Proposición", numerado por capítulo.
\newtheorem{lem}{Lema}[chapter]              % Define "Lema", numerado por capítulo.
\newtheorem{cor}{Corolario}[chapter]         % Define "Corolario", numerado por capítulo.
\theoremstyle{definition}                    % Cambia el estilo para definiciones y ejemplos.
\newtheorem{definition}{Definición}[chapter] % Define "Definición", numerado por capítulo.
\newtheorem{note}{Nota}[chapter]             % Define "Definición", numerado por capítulo.
\newtheorem{example}{Ejemplo}[chapter]       % Define "Ejemplo", numerado por capítulo.
\newtheorem{obs}{Observación}[chapter]       % Define "Observación", numerado por capítulo.

% Comandos para personalizar notación matemática
\renewcommand*{\deg}{\normalfont\text{gr}\hspace*{1mm}} % Redefine el símbolo de grados como "gr".
\renewcommand{\vec}[1]{\mathbf{#1}}                     % Define vectores como negritas.
\newcommand{\cl}[1]{\overline{#1}}                      % Define la clausura.
\newcommand{\conj}[1]{\overline{#1}}                    % Define conjugación.
\newcommand{\fr}[1]{\partial{#1}}                       % Define frontera.
\newcommand{\ip}[2]{\left\langle #1,#2\right\rangle}    % Define producto interno.
\newcommand{\norm}[1]{\left\| #1\right\|}               % Define norma.
\newcommand{\ec}[1]{\left[#1\right]}                    % Define corchetes.
\newcommand{\set}[1]{\left\lbrace#1\right\rbrace}       % Define conjuntos.
\newcommand{\suc}[1]{\left\lbrace#1\right\rbrace}       % Define sucesiones.
\newcommand{\abs}[1]{\left| #1\right|}                  % Define valor absoluto.
\newcommand{\ent}[1]{\left\lfloor #1\right\rfloor}      % Define parte entera.
\newcommand{\eps}{\varepsilon}                          % Redefine epsilon.
\renewcommand{\Im}{\operatorname{Im}}                   % Redefinición del operador "Parte imaginaria"
\renewcommand{\Re}{\operatorname{Re}}                   % Redefinición del operador "Parte real"
\DeclareMathOperator{\Log}{Log}                         % Operador "Logaritmo"
\DeclareMathOperator{\inter}{int}                       % Operador "Interior"
\DeclareMathOperator{\mcm}{mcm}                         % Operador "mínimo común múltiplo"
\DeclareMathOperator{\mcd}{mcd}                         % Operador "máximo común divisor"
\DeclareMathOperator{\Cov}{Cov}                         % Operador matemático "Covarianza"

% Notación para conjuntos numéricos
\def\NN{\mathbb{N}} % Números naturales
\def\ZZ{\mathbb{Z}} % Números enteros
\def\QQ{\mathbb{Q}} % Números racionales
\def\RR{\mathbb{R}} % Números reales
\def\CC{\mathbb{C}} % Números complejos

% Notación para vectores
\def\0{\vec{0}}
\def\1{\vec{1}}
\def\a{\vec{a}}
\def\b{\vec{b}}
\def\u{\vec{u}}
\def\p{\vec{p}}
\def\t{\vec{t}}
\def\x{\vec{x}}
\def\y{\vec{y}}
\def\z{\vec{z}}
\def\f{\vec{f}}
\def\g{\vec{g}}

% Documento principal
\begin{document}

% Portada
\title{\bf MATEMÁTICAS}                     % Título del libro
\author{}                                   % Autor del libro
\maketitle                                  % Genera la portada

% Índice
\immediate\tableofcontents                   % Genera el índice inmediatamente.

% Incluir partes del libro
\nocite{*}                                   % Incluye todas las referencias en la bibliografía, aunque no estén citadas.

\part{Análisis matemático}

\chapter{Límites y continuidad}

\section{Introducción}

Supongamos al lector ya familiarizado con el concepto de límites tal como es introducido en el cálculo elemental donde es corriente presentar varios tipos de límites. Por ejemplo, el {\it límite de una sucesión} de números reales $\suc{x_n}$, que simbolizamos cuando escribimos
$$\lim_{n\to\infty}x_n=A$$
significa que para cada número $\eps>0$ existe un entero $N>0$ tal que
$$\abs{x_n-A}<\eps\quad\text{siempre que }n\geq N$$
Este límite pretende transmitir la idea intuitiva de que $x_n$ puede estar suficientemente próximo a $A$ en el supuesto de que $n$ sea suficientemente grande. También se da el {\it límite de una función}, indicado por medio de la notación
$$\lim_{x\to p}f(x)=A$$
que significa que para cada $\eps>0$ existe otro número $\delta>0$ tal que
$$\abs{f(x)-A}<\eps \quad\text{siempre que }0<\abs{x-p}<\delta$$
Esta definición expresa la idea de que $f(x)$ puede conseguirse tan próximo a $A$ como queramos, siempre que $x$ se tome lo suficientemente próximo a $p$, sin llegar a ser $p$.

Las aplicaciones del cálculo a los problemas geométricos y físicos del espacio tridimensional y a las funciones de varias variables nos obligan a extender estos conceptos a $\RR^n$. Es tan necesario como fácil dar un paso más e introducir los límites en el marco más general de los espacios métricos. Esto simplifica la teoría puesto que elimina restricciones innecesarias y al mismo tiempo cubre casi todos los aspectos necesarios del análisis.

Primeramente discutiremos los límites de las sucesiones de puntos de un espacio métrico y después discutiremos los límites de funciones y el concepto de continuidad.

\section{Sucesiones convergentes en un espacio métrico}

\begin{definition}
    Una sucesión $\suc{x_n}$ de puntos de un espacio métrico $(S, d)$ es convergente si existe un punto $p$ de $S$ que satisfaga la siguiente propiedad:

    Para todo $\eps>0$ existe un entero $N$ tal que
    $$d(x_n,p)<\eps\quad\text{siempre que }n\geq N$$
    Diremos también que $\suc{x_n}$ converge hacia $p$ y escribiremos $x_n\to p$ cuando $n\to\infty$, o simplemente $x_n\to p$. Si no existe un tal número $p$ de $S$, se dice que la sucesión $\suc{x_n}$ es divergente.
\end{definition}

La definición de convergencia implica que
$$x_n\to p\quad\text{si, y sólo si, }d(x_n, p)\to 0$$
La convergencia de la sucesión $\suc{d(x_n, p)}$ hacia $0$ se realiza en el espacio euclídeo $\RR$.

\begin{example}
    En el espacio euclídeo $\RR$, una sucesión $\suc{x_n}$ se llama {\it creciente} si $x_n\leq x_{n+1}$ para todo $n$. Si una sucesión creciente está acotada superiormente (esto es, si $x_n\leq M$ para un $M>0$ y para todo $n$), entonces $\suc{x_n}$ converge hacia el supremo de su recorrido, $\sup\set{x_1,x_2,\dots}$. Análogamente, $\suc{x_n}$ se llama {\it decreciente} si $x_{n+1}\leq x_n$ para todo $n$. Cada sucesión decreciente acotada inferiormente converge hacia el ínfimo de su recorrido. Por ejemplo, $\suc{1/n}$ converge a $0$.
\end{example}

\begin{example}
    Si $\suc{a_n}$ y $\suc{b_n}$ son sucesiones reales que convergen hacia $0$, entonces $\suc{a_n+b_n}$ también converge hacia $0$. Si $0\leq c_n\leq a_n$ para todo $n$ y si $\suc{a_n}$ converge hacia $0$, entonces $\suc{c_n}$ también converge hacia $0$. Estas propiedades elementales de las sucesiones $\RR$ pueden ser útiles para simplificar algunas de las demostraciones concernientes a límites de un espacio métrico general.
\end{example}

\begin{example}
    En el plano complejo $\CC$, sea
    $$z_n=1+\frac{1}{n^2}+\left(2-\frac{1}{n}\right)i$$
    Entonces $\suc{z_n}$ converge hacia $1+2i$ puesto que
    $$d(z_n, 1+2i)^2=\abs{z_n - (1+2i)}^2=\abs{\frac{1}{n^2}-\frac{1}{n}i}^2=\frac{1}{n^4}+\frac{1}{n^2}\to 0\text{ cuando }n\to\infty$$
    luego $d(z_n, 1+2i)\to 0$.
\end{example}

\begin{theorem}
    Una sucesión $\suc{x_n}$ en un espacio métrico $(S,d)$ puede converger hace un punto de $S$, a lo sumo.
\end{theorem}

\begin{proof}
    Supongamos que $x_n\to p$ y que $x_n\to q$. Probaremos que $p=q$. En virtud de la desigualdad triangular se tiene
    $$0\leq d(p,q)\leq d(p,x_n)+d(x_n,q)$$
    Como $d(p,x_n)\to 0$ y $d(x_n,q)\to 0$ se tiene que $d(p,q)=0$, luego $p=q$.
\end{proof}

Si una sucesión $\suc{x_n}$ converge, el único punto  hacia el que converge se llama {\it límite} de la sucesión y se designa por medio de $\lim x_n$ o por medio de $\lim_{n\to\infty}x_n$.


\part{Cálculo Vectorial}

\chapter{Transformaciones lineales y matrices}

% \section{Transformaciones lineales}
% \section{Núcleo y recorrido}
% \section{Dimensión del núcleo y rango de la transformación}
% \section{Ejercicios}
% \section{Operaciones algebraicas con transformaciones lineales}
% \section{Inversas}
% \section{Transformaciones lineales uno a uno}
% \section{Ejercicios}
% \section{Transformaciones lineales con valores asignados}
\section{Representación matricial de las transformaciones lineales}

Ya que es posible obtener distintas representaciones matriciales de una transformación lineal dada mediante la elección de bases distintas, parece natural intentar elegir bases de modo que la matriz resultante tenga una forma lo más sencilla posible. El teorema que sigue prueba que podemos hacer todos los elementos $0$ excepto los de la diagonal que va desde el vértice superior izquierdo al inferior derecho. A lo largo de esa diagonal habrá una hilera de unos seguidos de ceros, siendo el número de unos igual al rango de la transformación. Una matriz $(t_{ik})$ con todos los elementos $t_{ik}=0$ cuando $i\neq k$ se llama \textbf{matriz diagonal}.

\begin{theorem}
    Sean $V$ y $W$ espacios lineales de dimensión finita, con $\dim V=n$ y $\dim W=m$. Supongamos que $T\in\mathcal{L}(V,W)$ y que $r=\dim T(V)$ representa el rango de $T$. Existen entonces una base $(e_1,\dots,e_n)$ para $V$ y otra $(w_1,\dots,w_m)$ para $W$ tales que
    \begin{equation}
        T(e_i)=w_i\quad\text{para } i=1,2,\dots,r
    \end{equation}
    y
    \begin{equation}
        T(e_i)=0\quad\text{para } i=r+1,\dots,n
    \end{equation}
    Por consiguiente, la matriz $(t_{ik})$ de $T$ relativa a esas bases tiene todos los elementos cero excepto los $r$ elementos de la diagonal que valen
    $$t_{11}=t_{22}=\dots=t_{rr}=1$$

\end{theorem}

\part{Probabilidad y estadística}

\chapter{Probabilidad}

\section{Espacio muestral}
En el estudio de la estadística, nos ocupamos fundamentalmente de la \textbf{presentación e interpretación de resultados aleatorios} que surgen en investigaciones planificadas o estudios científicos. Por ejemplo, en Estados Unidos, para justificar la instalación de un semáforo, se podría registrar el número de accidentes mensuales en la intersección de Driftwood Lane y Royal Oak Drive. En una fábrica, los artículos producidos en la línea de ensamblaje podrían clasificarse como ``defectuosos'' o ``no defectuosos''. En una reacción química, se podría medir el volumen de gas liberado al variar la concentración de un ácido. Así, quienes trabajan con estadística manejan tanto \textbf{datos numéricos}, que reflejan conteos o mediciones, como \textbf{datos categóricos}, que se clasifican según un criterio específico.

En este contexto, cualquier registro de información, ya sea numérico o categórico, se denominará \textbf{observación}. Por ejemplo, los números $2, 0, 1$ y $2$, que representan el número de accidentes ocurridos de enero a abril del año pasado en la intersección mencionada, constituyen un conjunto de observaciones. Del mismo modo, las categorías $N, D, N, N$ y $D$ donde $N$ indica ``no defectuos'' y $D$ ``defectuoso'' forman un conjunto de observaciones al inspeccionar cinco artículos.

El término \textbf{experimento} en estadística describe cualquier proceso que genere un conjunto de datos. Un ejemplo sencillo es el lanzamiento de una moneda, que produce dos posibles resultados: cara o cruz. Otro ejemplo es el lanzamiento de un misil, donde se observa su velocidad en tiempos específicos. Incluso las opiniones de votantes sobre un nuevo impuesto pueden considerarse observaciones de un experimento. En estadística, nos interesan particularmente las observaciones obtenidas al repetir un experimento varias veces. En la mayoría de los casos, los resultados están influenciados por el azar y, por tanto, no se pueden predecir con certeza. Por ejemplo, si un químico repite un análisis bajo las mismas condiciones, las medidas obtenidas variarán, evidenciando un elemento de probabilidad en el procedimiento. Del mismo modo, aunque lancemos una moneda al aire repetidamente, no podemos garantizar que obtendremos cara en un lanzamiento específico, aunque sí conocemos todas las posibilidades para cada lanzamiento.

En los tres tipos principales de estudios estadísticos —\textbf{diseños experimentales}, \textbf{estudios observacionales} y \textbf{estudios retrospectivos}— el resultado final es siempre un conjunto de datos, inevitablemente sujeto a \textbf{incertidumbre}. Aunque solo uno de ellos incluye la palabra ``experimento'' en su descripción, tanto el proceso de generar los datos como el de observarlos forman parte de un experimento.

\begin{definition}
    Al conjunto de todos los resultados posibles de un experimento estadístico se le llama \textbf{espacio muestral} y se representa por el símbolo $S$.
\end{definition}
\part{Análisis de Datos Multivariantes}

\chapter{Álgebra matricial}

\section{Definiciones básicas}

Un conjunto de $n$ números reales $\x$ puede representarse como un punto en el espacio de $n$ dimensiones $\RR^n$. Definiremos el vector $\x$ como el segmento orientado que une el origen de coordenadas con el punto $\x$. La orientación es importante, porque no es lo mismo el vector $\x$ que el $-\x$. Con esta correspondencia, a cada punto del espacio en $\RR^n$ le asociamos un vector. En adelante, representaremos un vector mediante $\x$, para diferenciarlo del escalar $x$, y llamaremos $\RR^n$ al espacio de todos los vectores de $n$ coordenadas o componentes. En particular, un conjunto de números con todos los valores iguales se representará por un vector \textbf{constante}, que es aquél con todas sus coordenadas iguales. Un vector constante es de la forma $c\1$, donde $c$ es cualquier constante y $\1$ es el vector con todas sus coordenadas iguales a la unidad.

En Estadística podemos asociar a los valores de una variable en $n$ elementos un vector en $\RR^n$, cuyo componente $i$-ésimo es el valor de la variable en el elemento $i$. Por ejemplo, si medimos las edades de tres personas en una clase y obtenemos los valores 20, 19 y 21 años, esta muestra se representa por el vector tridimensional
$$\x=\begin{pmatrix}
    20 \\
    19 \\
    21
\end{pmatrix}$$

La \textbf{suma} (o \textbf{diferencia}) de dos vectores $\x$ e $\y$, ambos en $\RR^n$, se define como un nuevo vector con componentes iguales a la suma (diferencia) de los componentes de los sumandos:
$$\x + \y=\begin{pmatrix}
    x_1 \\
    \vdots \\
    x_n
\end{pmatrix}+
\begin{pmatrix}
    y_1 \\
    \vdots \\
    y_n
\end{pmatrix} =
\begin{pmatrix}
    x_1 + y_1 \\
    \vdots \\
    x_n + y_n
\end{pmatrix}$$

Es inmediato comprobar que la suma de vectores es asociativa, $\x+(\y+\z)=(\x+\y)+\z$ y conmutativa, $\x+\y=\y+\x$.

La suma de dos vectores corresponde a la idea intuitva de trasladar un vector al extremo del otro y construir la línea que va desde el origen del primero al extremo del segundo. La operación suma (resta) de dos vectores da lugar a otro vector y estadísticamente corresponde a generar una nueva variable como suma (resta) de otras dos anteriores. Por ejemplo, si $\x$ representa el número de trabajadores varones en un conjunto de empresas e $\y$ el número de trabajadoras, la variable $\x+\y$ representa el número total de trabajadores y la variable $\x-\y$ la diferencia entre hombres y mujeres de cada empresa.

\textbf{El producto de una constante por un vector}, es un nuevo vector con componentes los del vector inicial multiplicados por la constante.
$$\z=k\x=\begin{pmatrix}
    kx_1 \\
    \vdots \\
    kx_n
\end{pmatrix}$$

Multiplicar por una constante equivale a un cambio en las unidades de medición. Por ejemplo, si en lugar de medir el número de trabajadores en unidades (variable $\x$) lo hacemos en centenas (variable $\z$), entonces la variable $\z$ es igual a $\x/100$.

Llamaremos \textbf{vector transpuesto} $\x'$, de otro $\x$, a un vector con las mismas componentes, pero escritas ahora en fila:
$$\x'=\left(x_1,\dots,x_n\right)$$

Al transponer un vector columna se obtiene un vector fila. Generalmente los vectores fila se utilizan para describir los valores de $p$ variables distintas en un mismo elemento de una población.

El \textbf{producto escalar o interno} de dos vectores $\x$, $\y$, ambos en $\RR^n$, que escribiremos $\x'\y$ o $\y'\x$, es el escalar obtenido al sumar los productos de sus componentes.
$$\x'\y=\y'\x=\sum_{i=1}^nx_iy_i$$

Se llamará \textbf{norma (cuadrática)} o longitud de un vector $\x$, a la raíz cuadrada positiva del producto escalar $\x'\x$. Se escribe $\norm{\x}$:
$$\norm{x}=\sqrt{\x'\x}=\sqrt{x_1^2+\cdots+x_n^2}$$

La norma es la longitud del segmento que une el origen con el punto $\x$.

El producto escalar puede calcularse también como el producto de las normas de los vectores por el coseno del ángulo que forman. Para ilustrar este concepto consideremos los vectores
$$\x=\begin{pmatrix}
    a \\
    0
\end{pmatrix}, \quad \y=\begin{pmatrix}
    a \\
    c
\end{pmatrix}$$
Observemos que el producto escalar es $\x'\y=a^2$ y que este mismo resultado se obtiene multiplicando la norma de ambos vectores, $\norm{\x}=a$, $\norm{\y}=\sqrt{a^2+c^2}$ por el coseno del ángulo $\theta$ que forman, dado por $\cos\theta=a/\sqrt{a^2+c^2}$. Observemos que el producto escalar puede también expresarse como el producto de la norma de un vector por la proyección del otro sobre él. Si uno de los vectores tiene norma uno, el producto escalar es directamente la proyección del otro vector sobre él.

Se demuestra en general que:
$$\abs{\x'\y}\leq\norm{\x}\norm{\y}$$
que se conoce como la \textbf{desigualdad de Cauchy-Schwarz}. Esta desigualdad permite definir el \textbf{ángulo} entre dos vecotres $\x$ e $\y$ de $\RR^n$ (no nulos) por la relación:
$$\cos\theta=\frac{\x'\y}{\norm{\x}\norm{\y}}$$

Si dos variables tienen media cero, el coseno del ángulo que forman es su \textbf{coeficiente de correlación}.

Dos vectores son \textbf{ortogonales}, o perpendiculares, si y sólo si, su producto escalar es cero. Por la definición de ángulo
$$\x'\y=\norm{\x}\norm{\y}\cos\theta$$
siendo $\theta$ el ángulo que forman los vectores. Si $\theta=90^\circ$ el coseno es cero y también lo será su producto escalar.

El producto escalar tiene una clara interpretación estadística. Para describir una variable tomamos su media. Para describir un vector podemos tomar su proyección sobre el vector constante. El vector constante de norma unidad en dimensión $n$ es $\frac{1}{\sqrt{n}}\1$, y la proyección de $\x$ sobre este vector es $\frac{1}{\sqrt{n}}\1'\x=\sum x_i/\sqrt{n}=\bar{x}\sqrt{n}$. El vector constante resultante de esta proyección es
$$\bar{x}\sqrt{n}\left(\frac{1}{\sqrt{n}}\right)\1=\bar{x}\1$$
Por tanto, la media es el escalar que define el vector obtenido al proyectar el vector de datos sobre la dirección constante. También puede interpretarse como la norma estandarizada del vector obtenido al proyectar los datos en la dirección del vector constante, donde para estandarizar la norma de un vector dividiremos siempre por $\sqrt{n}$, siendo $n$ la dimensión del espacio.

La variabilidad de los datos se mide por la \textbf{desviación típica}, que es la distancia estandarizada entre el vector de datos y el vector constante. La proyección del vector de datos sobre la dirección del vector constante produce el vector $\bar{x}\1$, y la norma del vector diferencia, $\x-\bar{x}\1$, mide la distancia entre el vector de datos y el vector constante. La norma estandarizada, dividiendo por la raíz de la dimensión del espacio es:
$$\frac{1}{\sqrt{n}}\norm{x-\bar{x}\1}=\sqrt{\frac{\sum_{i=1}^n(x_i-\bar{x})^2}{n}}=\sigma_\x$$

La medida de dependencia lineal entre dos variables, $\x$, $\y$ es la \textbf{covarianza}. La covarianza es el producto escalar promedio de los dos vectores medidos en desviaciones a la media, o tomando sus diferencias respecto a la proyección sobre el vector constante. Si promediamos el producto escalar de estos vectores
$$\frac{1}{n}(\x-\bar{x}\1)'(\y-\bar{y})=\frac{\sum_{i=1}^n(x_i-\bar{x})(y_i-\bar{y})}{n}=\Cov(\x,\y)$$
se obtiene la covarianza. Para variables con media cero, el producto escalar de los dos vectores, dividido por $n$, es directamente la covarianza.

Para variables estandarizadas (de media cero y desviación típica uno) la covarianza es el \textbf{coeficiente de correlación}:
$$\rho=\frac{\Cov(\x,\y)}{\sigma_\x\sigma_\y}=\Cov(\x,\y)$$

Para vectores media cero, el coeficiente de correlación es el coseno del ángulo entre los vectores que las representan:
$$\rho=\frac{\Cov(\x,\y)}{\sigma_\x\sigma_y}=\frac{\dfrac{\x'\y}{n}}{\dfrac{\norm{x}}{\sqrt{n}}\dfrac{\norm{y}}{\sqrt{n}}}=\frac{\x'\y}{\norm{x}\norm{y}}=\cos\theta$$
que es la interpretación geométrica del coeficiente de correlación. La implicación estadística de la ortogonalidad es la incorrelación. Si dos variables tienen media cero y son ortogonales, es decir, los vectores que las caracterizan forman un ángulo de 90 grados, $\rho =0$, las variables están incorreladas.







% Bibliografía
\bibliographystyle{apalike}                  % Usa el estilo APA para la bibliografía.
\phantomsection                              % Fija el punto de anclaje para el enlace desde el índice.
\addcontentsline{toc}{chapter}{Bibliografía} % Añade la bibliografía al índice de contenidos.
\bibliography{bib/references}                % Incluye la bibliografía del archivo references.bib.

\end{document}