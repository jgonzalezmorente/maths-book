\chapter{Probabilidad}

\section{Espacio muestral}
En el estudio de la estadística, nos ocupamos fundamentalmente de la \textbf{presentación e interpretación de resultados aleatorios} que surgen en investigaciones planificadas o estudios científicos. Por ejemplo, en Estados Unidos, para justificar la instalación de un semáforo, se podría registrar el número de accidentes mensuales en la intersección de Driftwood Lane y Royal Oak Drive. En una fábrica, los artículos producidos en la línea de ensamblaje podrían clasificarse como ``defectuosos'' o ``no defectuosos''. En una reacción química, se podría medir el volumen de gas liberado al variar la concentración de un ácido. Así, quienes trabajan con estadística manejan tanto \textbf{datos numéricos}, que reflejan conteos o mediciones, como \textbf{datos categóricos}, que se clasifican según un criterio específico.

En este contexto, cualquier registro de información, ya sea numérico o categórico, se denominará \textbf{observación}. Por ejemplo, los números $2, 0, 1$ y $2$, que representan el número de accidentes ocurridos de enero a abril del año pasado en la intersección mencionada, constituyen un conjunto de observaciones. Del mismo modo, las categorías $N, D, N, N$ y $D$ donde $N$ indica ``no defectuos'' y $D$ ``defectuoso'' forman un conjunto de observaciones al inspeccionar cinco artículos.

El término \textbf{experimento} en estadística describe cualquier proceso que genere un conjunto de datos. Un ejemplo sencillo es el lanzamiento de una moneda, que produce dos posibles resultados: cara o cruz. Otro ejemplo es el lanzamiento de un misil, donde se observa su velocidad en tiempos específicos. Incluso las opiniones de votantes sobre un nuevo impuesto pueden considerarse observaciones de un experimento. En estadística, nos interesan particularmente las observaciones obtenidas al repetir un experimento varias veces. En la mayoría de los casos, los resultados están influenciados por el azar y, por tanto, no se pueden predecir con certeza. Por ejemplo, si un químico repite un análisis bajo las mismas condiciones, las medidas obtenidas variarán, evidenciando un elemento de probabilidad en el procedimiento. Del mismo modo, aunque lancemos una moneda al aire repetidamente, no podemos garantizar que obtendremos cara en un lanzamiento específico, aunque sí conocemos todas las posibilidades para cada lanzamiento.

En los tres tipos principales de estudios estadísticos —\textbf{diseños experimentales}, \textbf{estudios observacionales} y \textbf{estudios retrospectivos}— el resultado final es siempre un conjunto de datos, inevitablemente sujeto a \textbf{incertidumbre}. Aunque solo uno de ellos incluye la palabra ``experimento'' en su descripción, tanto el proceso de generar los datos como el de observarlos forman parte de un experimento.

\begin{definition}
    Al conjunto de todos los resultados posibles de un experimento estadístico se le llama \textbf{espacio muestral} y se representa por el símbolo $S$.
\end{definition}