\chapter{Espacios lineales}

\section{Introducción}

A lo largo de la Matemática se encuentran muchos ejemplos de objetos matemáticos que pueden sumarse unos con otros y multiplicarse por números reales. Ante todo, los números reales son objetos de tal naturaleza. Otros ejemplos son las funciones vectoriales, los números complejos, las series y los vectores en el espacio $n$-dimensional. En este capítulo tratamos un concepto matemático general, llamado {\it espacio lineal}, que incluye todos esos ejemplos y muchos otros como casos particulares.

Brevemente, un espacio lineal es un conjunto de elementos de naturaleza cualquiera sobre el que pueden realizarse ciertas operaciones llamadas {\it adición} y {\it multiplicación por números}. Al definir un espacio lineal no especifacmos la naturaleza de los elementos ni decimos cómo se realizan las operaciones entre ellos. En cambio, exigimos que las operaciones tengan ciertas propiedades que tomamos como axiomas de un espacio lineal. Vamos ahora a hacer con detalle una descripción de esos axiomas.